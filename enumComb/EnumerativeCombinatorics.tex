\documentclass[12pt]{article}
\usepackage{setspace}
\usepackage{import}
\usepackage{xifthen}
\usepackage{pdfpages}
\usepackage{transparent}
\usepackage{float}
\usepackage{tabularx,environ,amsmath,amssymb}

\newcommand{\leg}[2]{\left(\frac{#1}{#2}\right)}
\newcommand{\dsp}{\displaystyle}
\newcommand{\mdp}[1]{\left(\text{mod }{#1}\right)}
\newcommand{\ind}[2]{\text{ind}_{#2}{#1}}
\newcommand{\mb}{\mathbb}

\title{Enumerative Combinatorics}

\author{Simo Ryu \\ cloneofsimo@gmail.com}

\date{Korea University --- \today}

\begin{document}
\maketitle

\section{Advanced Topics in Enumeration}

\subsection{Set Partitions}

In this section, we are interested in making Set partitions and their enumerations.
For a finite set $S$ that  $$ |S| = m , S = \{ 1,2,3,...,m\}$$
a partition of S is defined as collection of non - empty $A_j \subset S$
such that \begin{equation}
A_i \cap A_j \neq \phi \text{ for all $i,j$}
\end{equation}
\begin{equation}
  A_1 \cup A_2 \cup A_3 \cdots \cup A_n = S
\end{equation}

\textbb{exercise) } In how many ways can you make partition of $S = {1,2,3,4}$ ?



\end{document}
