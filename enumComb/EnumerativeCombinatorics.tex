\documentclass[12pt]{article}
\usepackage{setspace}
\usepackage{import}
\usepackage{xifthen}
\usepackage{pdfpages}
\usepackage{transparent}
\usepackage{float}
\usepackage{tabularx,environ,amsmath,amssymb}
\usepackage{amsthm}
\usepackage[utf8]{inputenc}
\usepackage[english]{babel}

\newtheorem{theorem}{Theorem}[section]
\newtheorem{corollary}{Corollary}[theorem]
\newtheorem{lemma}[theorem]{Lemma}
\newtheorem*{example}{Example}

\newcommand{\leg}[2]{\left(\frac{#1}{#2}\right)}
\newcommand{\dsp}{\displaystyle}
\newcommand{\mdp}[1]{\left(\text{mod }{#1}\right)}
\newcommand{\ind}[2]{\text{ind}_{#2}{#1}}
\newcommand{\mb}{\mathbb}

\title{Enumerative Combinatorics}

\author{Simo Ryu \\ cloneofsimo@gmail.com}

\date{Korea University, \texttt{MATH464} --- \today}

\begin{document}
\maketitle

\section{Advanced Topics in Enumeration}

\subsection{Set Partitions}

In this section, we are interested in making Set partitions and their enumerations.
For a finite set $S$ that  $$ |S| = m , S = \{ 1,2,3,...,m\}$$
a partition of S is defined as collection of non - empty $A_j \subset S$
such that \begin{equation}
A_i \cap A_j \neq \phi \text{ for all $i,j$}
\end{equation}
\begin{equation}
  A_1 \cup A_2 \cup A_3 \cdots \cup A_n = S
\end{equation}

\textbf{exercise} In how many ways can you make partition of $S = \{1,2,3,4\}$ ?

Definition: Stirling number of second kind
$$S(n,k) = S_{n,k}  = \text{\# of partitions of $S$ (with $|S|$ = n) into $k$ parts}$$
It isn't enough to just make trivial cases with words, therefore:
\begin{align*}
  S(0,0) &= 0 \\
  S(0,k) &= 0 \\
  S(n,k) &= 0 \text{ for $(k>n)$}
\end{align*}

By simply brute force, one can make the following simple table for Sterling's 2nd number.
\begin{center}
  \begin{tabular}{c|c c c c c}
    $n,k$ & 0 & 1 & 2 & 3 & 4 \\
    \hline
    0 & 1 \\
    1 & 0 & 1 \\
    2 & 0 & 1 & 1 \\
    3 & 0 & 1 & 3 & 1\\
    4 & 0 & 1 & 2 & 6 & 1 \\

  \end{tabular}
\end{center}

Meanwhile, for \textbf{ordinary} binomial coefficients, we have the \textbf{Pascal's Identity}
\[
\binom{n}{k} = \binom{n-1}{k-1} + \binom{n-1}{k}
\]
The analogous equation for Stirling's 2nd number is as follows;
\[
S(n,k) = S(n-1,k-1) + kS(n-1,k)
\]
The proof is pretty simple. Just casework either if element 1 is "alone" or not. Left as trivial

Meanwhile, it is valuable to ask the following question: How many functions from $X = [n]$ to $Y = [k] $ are there? The answers can vary as the condition of functions are undefined. If we simply count all the functions, it is obviously $k^n$. But how many injective functions are there? The answer is exactly same as the definition of \textbf{falling factorial}.

\begin{description}
  \item[Falling Factorial $(k)_n$ is :]
\end{description}
\[
\begin{cases}
  k(k-1)(k-2)(k-3) \cdots (k-(n-1)) &\text{if } k \geq n \\
  0 &\text{if } k \leq n
\end{cases}
\]

It is pretty clear that injective function is as above, bijective function is non - existence (unless $k = n$, in that case, the answer is $k!$), so it is our interest to calculate the number of surjective functions.

\textbf{The idea is to consider the inverse image of each element in $Y$. It can be seen as set partition of $X$, and allowing permutation within the partitions.} Just like that, we have the following theorem.

\begin{theorem}[Number of Surjective Functions]
  The number of surjective functions from $X = [n]$ to $Y = [k]$ is $k!S(n,k)$.
\end{theorem}
 Again, proof is talaaeftr. (trivial and left as an exercise for the reader)
 More theorems are here.
\begin{theorem} We have for all $n,m\in \mb{N}$,
\[
S(m,n) = \sum_{j=0}^{m-1} \binom{m-1}{j} S(j,n-1)
\]
\end{theorem}
\begin{proof}
  We will view the partition of $[m]$ by considering the position of element $m$ and the size of the set that has it. Without loss of generality, let $m \in A_1$ and $|A_1| = m -j-1$. But now see that the left case for the specific assumption that we had above is simply partitioning $j$ elements into $n-1$ sets. It is quite obvious that therefore it is left to choose $m-j-1$ elements from $m-1$ elements.

\end{proof}

As one can see, when talking about functions, surjective or injective, one has to talk about \textbf{falling factorials, exponentials and Stirling's 2nd numbers}. This all comes down to following beautiful theorem:
\begin{theorem}
\[
n^m = \sum_{j=0}^{n} S(m,j)(n)_j
\]
\end{theorem}

\begin{example}[case for 2, 3]

\begin{align*}
x^2 &= S(2,0) + S(2,1)x + S(2,2)x(x-1) \\
x^3 &= S(3,0) + S(3,1)x + S(3,2)x(x-1) + S(3,3)x(x-1)(x-2)
\end{align*}

\end{example}

We will prove the case for $m\geq n$. As we've already mentioned, analogy of functions are necessary to prove this formula.
\begin{proof}
  $LHS := \text{number of functions from [m] to [n]}$

$RHS$:
\begin{figure}[H]
	\centering
	\def\svgwidth{\columnwidth}
	\import{./figures/}{func1.pdf_tex}
	\caption{Function from $[m]$ to $[n]$ goes to $J\subset Y$}
	\label{fig:func1}
\end{figure}
Here, $|J| = j$, and per every choice of $j$, there are $j! \binom{n}{j}$ choices. Per every set $J$, there are $S(m,j)$ choices of surjective functions.
Therefore per every $j$, there are $S(m,j) (n)_j$ many functions.
\end{proof}

\end{document}
