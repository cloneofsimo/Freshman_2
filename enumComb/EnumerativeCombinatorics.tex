\documentclass[12pt]{article}
\usepackage{setspace}
\usepackage{import}
\usepackage{xifthen}
\usepackage{pdfpages}
\usepackage{transparent}
\usepackage{float}
\usepackage{tabularx,environ,amsmath,amssymb}


\newcommand{\leg}[2]{\left(\frac{#1}{#2}\right)}
\newcommand{\dsp}{\displaystyle}
\newcommand{\mdp}[1]{\left(\text{mod }{#1}\right)}
\newcommand{\ind}[2]{\text{ind}_{#2}{#1}}
\newcommand{\mb}{\mathbb}

\title{Enumerative Combinatorics}

\author{Simo Ryu \\ cloneofsimo@gmail.com}

\date{Korea University, \texttt{MATH464} --- \today}

\begin{document}
\maketitle

\section{Advanced Topics in Enumeration}

\subsection{Set Partitions}

In this section, we are interested in making Set partitions and their enumerations.
For a finite set $S$ that  $$ |S| = m , S = \{ 1,2,3,...,m\}$$
a partition of S is defined as collection of non - empty $A_j \subset S$
such that \begin{equation}
A_i \cap A_j \neq \phi \text{ for all $i,j$}
\end{equation}
\begin{equation}
  A_1 \cup A_2 \cup A_3 \cdots \cup A_n = S
\end{equation}

\textbf{exercise} In how many ways can you make partition of $S = \{1,2,3,4\}$ ?

Definition: Stirling number of second kind
$$S(n,k) = S_{n,k}  = \text{\# of partitions of $S$ (with $|S|$ = n) into $k$ parts}$$
It isn't enough to just make trivial cases with words, therefore:
\begin{align*}
  S(0,0) &= 0 \\
  S(0,k) &= 0 \\
  S(n,k) &= 0 \text{ for $(k>n)$}
\end{align*}

By simply brute force, one can make the following simple table for Sterling's 2nd number.
\begin{center}
  \begin{tabular}{c|c c c c c}
    $n,k$ & 0 & 1 & 2 & 3 & 4 \\
    \hline
    0 & 1 \\
    1 & 0 & 1 \\
    2 & 0 & 1 & 1 \\
    3 & 0 & 1 & 3 & 1\\
    4 & 0 & 1 & 2 & 6 & 1 \\

  \end{tabular}
\end{center}

Meanwhile, for \textbf{ordinary} binomial coefficients, we have the \textbf{Pascal's Identity}
\[
\binom{n}{k} = \binom{n-1}{k-1} + \binom{n-1}{k}
\]
The analogous equation for Stirling's 2nd number is as follows;
\[
S(n,k) = S(n-1,k-1) + kS(n-1,k)
\]
The proof is pretty simple. Just casework either if element 1 is "alone" or not. Left as trivial

Meanwhile, it is valuable to ask the following question: How many functions from $X = [n]$ to $Y = [k] $ are there? The answers can vary as the condition of functions are undefined. If we simply count all the functions, it is obviously $k^n$. But how many injective functions are there? The answer is exactly same as the definition of \textbf{falling factorial}.

\begin{description}
  \item[Falling Factorial $(k)_n$ is :]
\end{description}
\[
\begin{cases}
  k(k-1)(k-2)(k-3) \cdots (k-(n-1)) &\text{if } k \geq n \\
  0 &\text{if } k \leq n
\end{cases}
\]

It is pretty clear that injective function is as above, bijective function is non - existence (unless $k = n$, in that case, the answer is $k!$), so it is our interest to


\end{document}
